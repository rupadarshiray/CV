\documentclass{LoLaTeXcv}
\usepackage[T1]{fontenc}
\usepackage[utf8]{inputenc}
\usepackage{amsfonts}

\usepackage{hyperref}
\hypersetup{
    colorlinks=true,
    linkcolor=blue,
    filecolor=magenta,      
    urlcolor=blue,
    pdftitle={Rupadarshi Ray}
    }


\begin{document}

\lltxPersonalInfo{
	Rupadarshi Ray}{
	\href{https://rupadarshiray.github.io}{rupadarshiray.github.io} \\
	%\href{https://twitter.com/rupadarshiray}{Twitter @rupadarshiray}
}{
	Fifth year mathematics major \\
	IISER Mohali}{
	% +91 9002469361 \\
	%\href{mailto:ms21165@iisermohali.ac.in}{ms21165@iisermohali.ac.in}\\
	\href{mailto:rupadarshi2014@gmail.com}{rupadarshi2014@gmail.com}}

%\lltxSection{Background}

\lltxSection{Education}

\

\begin{lltxJob}[5 years BSMS]{
		MS21, IISER Mohali}{
		2021-}{
		Mohali, Punjab}
	\item current CPI: 8.42
\end{lltxJob}

% \begin{lltxJob}[4 years CSE]{
%       Computer Sciences and Engineering, VIT Vellore}{
%       2020-2021}{
%             Vellore}
% \item (dropped-out) 
% \end{lltxJob}

\begin{lltxJob}[Primary, Middle and High Scool]
	{
		Techno India Group Public School (TIGPS), Balurghat}{
		2007-2020}{
		Balurghat, West Bengal}
	\item Standard XII CBSE board exams: 93.6\%
	\item Standard X CBSE board exams: 90.6\%
\end{lltxJob}

% \begin{lltxDescription}
% 	\item[2024-2025] Post doc., supervised by Jean-Pierre Serre \lltxdotfill
% 		Collège de Rance
% 	\item[2020-2023] PhD Mathematics, supervised by Alexander Grothendieck \lltxdotfill UCLA
% 	\item[2019-2020] BSc Mathematics \lltxdotfill Duketon University
% 	\item[2015-2018] MSc Mathematics \lltxdotfill Duketon University
% \end{lltxDescription}

\lltxSection{Thesis}

\

\begin{lltxJob}[Masters thesis under Dr. Arghya Mondal]{\href{https://dub.sh/thesisms}{Rigidity of locally symmetric spaces}}{Fall 2025-Spring 2026}{ IISER Mohali}
	\item Mostow rigidity theorem for locally symmetric spaces of non-compact type
\end{lltxJob}



\lltxSection{Academic interests}

\

My primary interests lie in

\begin{lltxItemize}
	\item Differential geometry: Riemannian manifolds, symplectic manifolds and Lie groups
	\item Dynamics on manifolds: Hamiltonian flows, geodesic flows and ergodic theory
	\item Symmetric spaces of non-compact type, their boundaries and rigidity of locally symmetric spaces
	% and lattices in semi-simple Lie groups
	% \item Dynamical systems and ODEs
	% \item Algebraic topology and differential geometry
	% \item Symplectic geometry
	% \item Representation theory
	% \item Fourier and harmonic analysis
	% \item Mathematical physics
\end{lltxItemize}
% \begin{lltxJob}[CSE core courses]{
%       Mathematics, Computer sciences}{
% 		First year of CSE}{
% 		VIT, Vellore}
% 	\item Compose hundreds of cantatas.
% 	\item Play the organ every sunday for the mass.
% \end{lltxJob}

I have done coursework and reading/also am interested in the following areas


\begin{lltxItemize}
	\item Symplectic geometry and Hamiltonian flows
	\item Representation theory of Lie groups and Lie algebras
	% \item Hyperbolic surfaces and Teichmuller theory
	\item Riemann surfaces, complex manifolds and complex analytic geometry
\end{lltxItemize}

% However, I am also broadly interested in the following

% \begin{lltxItemize} 
% 	\item 
% 	\item Integrable systems
% \end{lltxItemize}


\lltxSection{Talks}
\

\begin{lltxItemize}
	\item A talk on \textbf{irrational rotations on the torus} titled ``Are you dense in chaos?'' for the mathematics club of IISER Mohali, Infinity, in Spring 2024
	\item A talk on \textbf{some comparison theorems in Riemannian geometry} for the course on Riemannian geometry, IISER Mohali, Semester 7
	\item A talk on \textbf{knower's and liar's paradoxes and their formalizations} titled ``What does a knower and a liar have in common?'', course on Philosophy of language, IISER Mohali, Semester 7
	\item A short talk on \textbf{topological conjuacy of linear flows} for the course on seminar delivery, IISER Mohali, Semester 7
	\item A talk on \textbf{periods of elliptic curves} for the course on Arithmetic of elliptic curves, IISER Mohali, Semester 8
	\item A short talk on \textbf{construction of Riemann surface of holomorphic functions} for the course on seminar delivery, IISER Mohali, Semester 8
	\item A talk on \textbf{construction of Haar measure on locally compact Hausdorff groups} for the course on Fourier analysis, IISER Mohali, Semester 8
	\item \textbf{Symplectic manifolds and Hamiltonian flows}
		\subitem \href{https://ggl.link/hamiltonian}{Introductory talk titled ``When
		does a vector field kill area?''} in the \href{https://web.iisermohali.ac.in/dept/math/seminars/gsg/}{Graduate Students Seminar, IISER Mohali}, Spring 2025
		\subitem \href{https://dub.sh/qofhamil}{Talk with regards to the orbit method in representation theory}, Summer 2025
		% \subitem \href{https://dub.sh/symplectictoric}{Talks towards symplectic toric manifolds} in a discord server
	\item \textbf{Abel's theorem and Jacobi's inversion theorem}, Meromorphic functions on Riemann surfaces seminar, IISER Mohali, Summer 2025
	\item \textbf{Ergodic theorems for actions of locally compact groups}, Summer reading presentation, IISER Mohali, Summer 2025
	\item \textbf{Reductive subgroups of $GL(n,\mathbb{R})$ and totally geodesic submanifolds of $P(n,\mathbb{R})$}, Boundary of symmetric spaces seminar, IISER Mohali, Fall 2025
\end{lltxItemize}


\lltxSection{Background}

\lltxSubSection{First and second year of BSMS}

\begin{lltxJob}[Personal reading from lectures by V Balakrishnan]{Ordinary differential equations, Hamiltonian dynamics and quantum mechanics}{Summer 2022}{}
	\item Classification of linear flows in $\mathbb{R}^2$
	\item Some theorems in Hamiltonian dynamics
	\item Quantum mechanical systems as a representation of the Hisenberg Lie algebra
\end{lltxJob}

\begin{lltxJob}[Mathematics course instructed by Dr. Shane D' Mello]{Differential geometry of curves and surfaces}{Semester 3}{IISER Mohali}
	\item Curvature of curves and surfaces in $\mathbb{R}^3$
	\item Gauss's theorema egregium
\end{lltxJob}

\begin{lltxJob}[Undergraduate physics camp]{
		\href{https://rupadarshiray.github.io/academicmatters/NIUS-19.1/NIUS-19.1-36-Rupadarshi-Ray-certificate.pdf}{NIUS 19.1, HBCSE-TIFR}}{
		December 2022}{
		HBCSE-TIFR, Mumbai}
	\item Attended talks about quantum computation, history of phase transitions, phase transitions in Ising model and neural networks, etc.
	\item Attended laboratory sessions "experimental problem solving".
\end{lltxJob}

\begin{lltxJob}[Personal reading]{Algebraic topology and smooth manifolds}{Spring 2023}{}
	\item Fundamental group, singular homology, categories and functors
	\item Differential forms (derivative and integrals) and vector fields (bracket and flows) on manifolds
\end{lltxJob}

\begin{lltxJob}[Summer reading under Dr. Shane D' Mello]{Sheaf theoretic proof of de Rham isomorphism}{Summer 2023}{IISER Mohali}
	\item Lee on de Rham cohomology of differential forms
	\item Griffiths and Harris on sheaf cohomology of differential forms
	\item Bott and Tu on Cech-de Rham double complex
\end{lltxJob}

\lltxSubSection{Third and fourth year of BSMS}

\begin{lltxJob}[Reading under Dr. Vaibhav Vaish]{Representation theory of groups and Lie algebras}{Winter 2023-Summer 2024}{}
	\item Fulton and Harris on representation theory of finite groups and Lie algebras
	\item Woit on quantum mechanics and representation theory
\end{lltxJob}

% \begin{lltxJob}[Mathematics course instructed by Dr. Kapil Hari Paranjape]{Complex analysis}{Semester 6}{IISER Mohali}
% 	\item Etale space of sheaf of holomorphic functions	
% \end{lltxJob}

\begin{lltxJob}[Mathematics course instructed by Dr. Shane D' Mello]{Knots and braids}{Semester 6}{IISER Mohali}
	\item Jones polynomial of knots and links
	\item Universal Abelian cover of knot complements and Alexander polynomial
	% \item Representations of Braid groups and Jones polynomial
\end{lltxJob}

\begin{lltsJob}[Mathematics course instructed by Dr. Soma Maity]{Riemannian geometry}{Semester 7}{IISER Mohali}
	% \item Riemannian manifolds, Levi-Civita connection and geodesics
	% \item Classification of constant curvature Riemannian manifolds
\end{lltsJob}

\begin{lltsJob}[Mathematics course instructed by Dr. Abhik Ganguli]{Arithmetic of elliptic curves}{Semester 8}{IISER Mohali}
\end{lltsJob}

\begin{lltxJob}[Seminar organized with talks by Dr. Kapil Hari Paranjape]{Meromorphic functions on Riemann surfaces}{Summer 2025}{IISER Mohali}
	\item Etale space of sheaf of holomorphic and meromorphic functions, ringed space definition of Riemann surfaces
	\item Constructing meromorphic functions on Riemann surfaces
	\item Riemann-Roch theorem, Abel's theorem and Jacobi's inversion theorem
\end{lltxJob}

\begin{lltxJob}[Workshop organized by Dr. Krishnendu Gongopadhyay and Dr. Pranab Sardar]{
	\href{https://docs.google.com/document/d/18rjLGn7hJHEmRk-QYcSysw5hH7nbYy5A}{Rigidity of Discrete Groups}
	}{June 30 - July 4, 2025}{IISER Mohali}
	\item Attended talks on the Mostow rigidity theorem for hyperbolic 3-manifolds by Dr. Arghya Mondal
\end{lltxJob}

\begin{lltxJob}[Summer reading under Dr. Jotsaroop Kaur]{Ergodic theorems for actions of locally compact groups}{Summer 2025}{IISER Mohali}
	\item Following Einsiedler and Ward, I presented proofs of ergodic theorems for measure preserving transformations and actions of Amenable groups with tripling property
\end{lltxJob}

\lltxSubSection{Fifth year of BSMS}

\begin{lltxJob}[Seminar organized by Dr. Arghya Mondal]{Boundary of symmetric spaces}{Fall 2025}{IISER Mohali}
	\item Symmetric spaces of non-compact type
	\item $P(n,\mathbb{R})$ and its totally geodesic submanifolds
\end{lltxJob}


% \lltxSection{Skills}

% \

%\lltxSubSection{Computer skills}
% \begin{lltxDescription}
% 	\item[Programming] Python, C, C++
% 	\item[Computational] MATLAB, R
% 	\item[Design] HTML, CSS, Adobe Illustrator
% \end{lltxDescription}


% \lltxSubSection{Practical skills}
% \begin{lltxDescription}
% 	\item[Languages] Body language, English, Lojban, the universal language of Love.
% 	\item[Mobility] Bike, car, fake taxi, helicopter, jet, motorcycle, submarine.
% 	\item[Other] People person person.
% \end{lltxDescription}

% \lltxSubSection{Passions}
% \begin{lltxDescription}
% 	\item[Sports] Chess (especially knight), ride my pony, aquapony.
% 	\item[Music] Good ol' Franco-Flemish School music, John Frusciante, Josh Klinghoffer, not Robert Fripp.
% 	\item[Series] $\sum_{n=1}^\infty n = -\frac{1}{12}$.
% \end{lltxDescription}


\lltxSection{Involvement}

% \lltxSubSection{}
\

\begin{lltxItemize}
	\item \textbf{Electronics project in Standard XII:} Implementation of A $\cdot$ B + C using resistance-transistor logic
	\item \href{https://rupadarshiray.github.io/academia/inculcation}{Wrote a route-map for introductory resources for mathematics and physics.}
	\item Volunteered in official help sessions on linear algebra for first year students in IISER Mohali in Monsoon 2023
	\item Volunteered to give a talk on \href{https://ggl.link/fourier}{Fourier series} in a summer camp in IISER Mohali
\end{lltxItemize}




% \begin{lltxItemize}
% 	\item Creator of the \emph{Association des Collaborateurs de Nicolas Bourbaki} in
% 		1952. The group's purpose was originally to write a rigorous textbook
% 		in analysis, but it grew to include presentations of many branches of
% 		algebra and analysis, including topology, from an axiomatic point of
% 		view. It eventually turned into a Tupperware party ; Grothendieck left it 
% 		over an agreement regarding the optimal tupperware shape.
% \end{lltxItemize}

% \lltxSubSection{Achievements}
% \begin{lltxItemize}
% 	\item Writer of fantasy story aimed at children, with national success — Judea, 33 B.C.
% 	\item First and only person to define what a set is in a Set theory book.
% \end{lltxItemize}

\end{document}
