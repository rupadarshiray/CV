\documentclass[21pt]{letter}
\usepackage[papersize={5in,7in}, total={4in,6in}]{geometry}
\usepackage{amsfonts}

\begin{document}
\begin{letter}{Statement of Purpose\\
Rupadarshi Ray }

\opening{}

\textbf{First and second years of BSMS}

When I came to the IISER Mohali campus after completing the first
semester of my BSMS online, I had some idea what groups and vector
spaces were. The first semester course on mathematics called
``symmetry'' ended on the definition of symmetry of a subset of
\(\mathbb{R}^{n}\) as the group of rotations, reflections and
translation that keep it stable. However, I had heard the term
\emph{symplectic manifold}, eluding to the subject of geometric
mechanics, or more generally topological dynamics on manifolds. Of
course the definition of \emph{manifolds}, \emph{Riemannian} or
\emph{symplectic manifolds} did not make any sense at the time. It was
at this time I opened a book on geometric mechanics in the library at
IISER Mohali saying \emph{torque free motion of a rigid body is the
geodesic flow (of a left invariant metric) on \({SO}(3)\)}.

The second semester course was on analysis on \(\mathbb{R}\), and
throughout the course I found myself relearning all the intuitive
notions about \emph{functions} I had after learning calculus. The
subjects of linear algebra and real analysis seemed to collide in the
definition of \emph{total derivative} of a multivariable function. I was
also trying to learn about \emph{differential forms} around this time,
having heard it was a ``proper'' was to integrate on higher dimensional
spaces.

Next, we had a course on Gauss's geometry of curves and surfaces. We
defined curvature of curves and surfaces in \(\mathbb{R}^{3}\) and I
found myself loving to learn and teach differential geometry to my
peers! I truly understood the notion of \emph{total derivative} as a
\emph{map between tangent spaces that sends velocities to velocities}
clearly in the setup of smooth map between surfaces.

After a semester on surfaces, I found the definition of manifolds and
smooth maps between them quite simple! Me and and a friend were looking
at courses in mathematics we can audit for fun. Given we knew the
definition of topological spaces and continuous maps between them, I
suggested we look at what happens in the course on \emph{algebraic
topology}. After a few classes of looking at homology and homotopy
groups and some ``ugly'' topological spaces, I started to really
appreciate the definition of manifolds. 

That summer, I luckily became a part of a reading group where we proved 
the de Rham isomorphism using
sheaf/Cech cohomology theory from Griffiths and Harris, Bott and Tu,
respectively. The worlds of \emph{differential forms} and
\emph{algebraic topology} collided in this wonderful isomorphism. Even
though I was young in doing mathematics, with practice and some
blackboxes I had gained some skill in chasing a diagram of vector spaces
and a background on the theory of smooth manifolds.

\textbf{Third and fourth years of BSMS}

After this, I began to understand the structure of vector fields and
their flows, which are fundamental to the theory of Lie groups, the
representation theory of Lie groups, and symplectic geometry. I could
now understand what \emph{geodesic flow on \({SO}(3)\)} meant that I
heard about more than a year ago! This is the point where I had my first
course on complex analysis. Thus, I could appreciate the topological,
homological and Lie theoretic aspects of the first course on complex analysis. Among other things, I
discovered the motivation behind sheaf cohomology by the Mitag-Leffler
problem on Riemann surfaces (from Griffiths and Harris, parts of which I
already read for sheaf theoretic de Rham isomorphism, but somehow missed
this section!).

The course ended with the construction of Etale space to
define global holomorphic functions. This led me to discover the
``Riemann surface of a (germ of a) holomorphic function''. After a back
and forth with this definition for a year, I started to understand what
the topology on the Etale space is \emph{doing}.

Latter in the summer, I spent some time studying representations of finite and Lie groups from Fulton and Harris' book.

\textbf{Retrospection on my training}

Over the course of my training, I have explored diverse areas of mathematics, including differential geometry, dynamical systems, symplectic geometry, complex geometry, representation theory, Lie groups, ergodic theory and mathematical physics. I have had the opportunity to credit courses on knot theory, Riemannian geometry, elliptic curves, Fourier analysis in my major years at IISER Mohali. Still, I believe the course I benefitted from the most was the one on curves and surfaces in third semester, which led to my interest in the theory of manifolds and algebraic topology quite early.

Engaging in above mentioned subjects led to me inculcate a diverse set of theoretical and problem solving skills, which firstly lets me engage in different conversations quickly and furthermore benefits me from having fruitful conversations with my peers in the mathematics community. I have had fun giving talks for the mathematics club, courses, graduate seminar etc.

\textbf{Fifth year of BSMS (masters thesis)}

After crediting the course on Galois theory and with the prior knowledge
of the theory of covering spaces, I started to appreciate the functorial
correspondence of topological theory of ramified coverings of compact
Riemann surfaces and Galois theory of field extensions of
\(\mathbb{C}(z)\). I wish to study classification/moduli spaces of such
ramified coverings for a fixed ramification profile. 

On the other side,
after crediting the course on Riemannian geometry (thus looking at the
hyperbolic space), I came to know about the Teichmuller and moduli space
of these structures on topological surfaces. Further, learning
about the history of elliptic and Abelian integrals, period of elliptic
curves, leading to the Jacobian of a Riemann surface, and the theorem of
Torelli gave me another map from the moduli space of Riemann surfaces to
\emph{some} space of lattices in \(\mathbb{C}^{g}\). This is another
result I wish to prove someday.

Latter in the summer, I attended a workshop on the \emph{rigidity of
discrete groups} held in IISER Mohali. I was fascinated by the theorems
on rigidity of hyperbolic 3-manifolds and some complex manifolds, having seen the theory of \emph{deformations} in dimension 2. 

Eventually, I decided to do my masters thesis on topics related to the strong rigidity of locally symmetric spaces: boundary of (globally) symmetric spaces and application of ergodic theory in this area. I became a part of a seminar on the same which has been a new experience for me! I hope to have another fun semester reading and writing for my masters thesis.

\textbf{Future prospects}

I wish to study topics related to (but not restricted to) 

\begin{itemize}
      \item discrete subgroups of Lie groups, 
      \item rigidity theory,
      \item arithmetic lattices and arithmetic manifolds,
      \item phenomena in non-positive curvature broadly, and
      \item rigidity and deformations of complex manifolds
\end{itemize}

however I am interested in geometry, topology and dynamics in general. I wish to join a graduate school in mathematics and eventually contribute to mathematics meaningfully, alongside studying the unread topics that left me fascinated.

\closing{
Rupadarshi Ray\\
Mathematics major\\
IISER Mohali}

\end{letter}
\end{document}